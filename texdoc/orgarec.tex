%%%%%%%%%%%%%%%%%%%%%%%%%%%%%%%%%%%%%
% Preamble for defining document properties
%%%%%%%%%%%%%%%%%%%%%%%%%%%%%%%%%%%%%
%%%%%%%%%%%%%%%%%%%%%%%%%%%%%%%%%%%%%
% Template for LCA-EPFL Students Report
%%%%%%%%%%%%%%%%%%%%%%%%%%%%%%%%%%%%%

\documentclass[11pt, a4paper, twoside]{report}

%%%%%%%%%%%%%%%%%%%%%%%%%%%%%%%%%%%%%
% Packages
%%%%%%%%%%%%%%%%%%%%%%%%%%%%%%%%%%%%%
\usepackage[applemac]{inputenc}
\usepackage{graphicx}
\usepackage{fullpage}
\usepackage{epsfig}
\usepackage{listings}
\usepackage{amsmath}
\usepackage{subfigure}
\usepackage{nomencl}
% For customizing header and footers
\usepackage{fancyhdr}
\usepackage{lscape}
\usepackage{algorithm}
\usepackage{algorithmic}
\usepackage{color}
\usepackage{dirtree}
\usepackage{hyperref}
\usepackage{multicol}

%%%%%%%%%%%%%%%%%%%%%%%%%%%%%%%%%%%%%
% Customizing headers and footers
%%%%%%%%%%%%%%%%%%%%%%%%%%%%%%%%%%%%%
\pagestyle{fancyplain}

%\fancypagestyle{plain}{  }

\renewcommand{\headrulewidth}{0pt}
\newcommand{\tit}{}
\newcommand{\aut}{FirstName LastNAme}
\renewcommand{\sectionmark}[1]{ \markright{\thesection.\ #1}}

% Header
\rhead[\fancyplain{}{} ] {\textbf{\fancyplain{ \nouppercase{\rightmark} \hspace{1cm} \thepage } { \nouppercase{\rightmark}\hspace{1cm} \thepage } }}
\chead[\fancyplain{}{} ] {\fancyplain{}{} }
\lhead[\textbf{\fancyplain{\thepage \hspace{1cm} \nouppercase{\leftmark} }{\thepage \hspace{1cm} \nouppercase{\leftmark}} }]{\fancyplain{}{} }
%Footer
\lfoot[\fancyplain{\tit}{\tit}] {\fancyplain{}{}}
\cfoot[\fancyplain{}{}] {\fancyplain{}{}}
\rfoot[\fancyplain{}{}] {\fancyplain{\tit}{\tit}}

\setlength{\headsep}{0.6cm}

%%%%%%%%%%%%%%%%%%%%%%%%%%%%%%%%%%%%%
% For code insertion
%%%%%%%%%%%%%%%%%%%%%%%%%%%%%%%%%%%%%
% "define" Scala
\lstdefinelanguage{scala}{
  morekeywords={abstract,case,catch,class,def,%
    do,else,extends,false,final,finally,%
    for,if,implicit,import,match,mixin,%
    new,null,object,override,package,%
    private,protected,requires,return,sealed,%
    super,this,throw,trait,true,try,%
    type,val,var,while,with,yield},
  otherkeywords={=>,<-,<\%,<:,>:,\#,@},
  sensitive=true,
  morecomment=[l]{//},
  morecomment=[n]{/*}{*/},
  morestring=[b]",
  morestring=[b]',
  morestring=[b]"""
}


\definecolor{dkgreen}{rgb}{0,0.6,0}
\definecolor{gray}{rgb}{0.5,0.5,0.5}
\definecolor{mauve}{rgb}{0.58,0,0.82}
 
% Default settings for code listings
\lstset{frame=tb,
  language=scala,
  aboveskip=3mm,
  belowskip=3mm,
  showstringspaces=false,
  columns=flexible,
  basicstyle={\small\ttfamily},
  numbers=none,
  numberstyle=\tiny\color{gray},
  keywordstyle=\color{blue},
  commentstyle=\color{dkgreen},
  stringstyle=\color{mauve},
  frame=single,
  breaklines=true,
  breakatwhitespace=true
  tabsize=3
}

\makeglossary
\renewcommand{\nomname}{List of Acronyms}

%makeindex PDMReport.glo -s nomencl.ist -o PDMReport.gls


% begin latex document
\begin{document}

%%%%%%%%%%%%%%%%%%%%%%%%%%%%%%%%%%%%%
% Title page
%%%%%%%%%%%%%%%%%%%%%%%%%%%%%%%%%%%%%
\newcommand{\logoepfl}[0]{\vspace*{1.5cm}\begin{center}\includegraphics[scale=0.4,keepaspectratio=true]{images/EPFL.pdf}\end{center}\hrule\vspace{1cm}}

\newcommand{\project}[1]{\begin{center}\large{#1}\end{center}}
\renewcommand{\title}[1]{\vspace{0.01cm}\begin{center}\LARGE{#1}\end{center}\vspace{0.2cm}}
\renewcommand{\author}[1]{\begin{center}\Large{#1}\end{center}}
\newcommand{\department}[1]{\begin{center}\large{#1}\end{center}}
\renewcommand{\date}[2]{\begin{center}\normalsize{#1 #2}\end{center}}

\newcommand{\supervisor}[4]{\begin{center}\begin{normalsize}{\bf #1}\\#2\\#3\\#4\end{normalsize}\end{center}}


\thispagestyle{empty}

\logoepfl

\project{Studyplan Visualisation}
\title{Project Documentation}
\author{Christopher Chiche \\ Julien Perrochet}
\department{CRAFT \\ EPFL}
\date{\today}

\begin{center}
\begin{tabular}{cc}
\begin{tabular}{p{6cm}}
\supervisor{Supervisor}{Pierre Dillenbourg}{EPFL}{CH-1015, Lausanne}
\end{tabular}
\end{tabular}
\end{center}

\clearpage \thispagestyle{empty} \cleardoublepage


%%%%%%%%%%%%%%%%%%%%%%%%%%%%%%%%%%%%%
% Dedication page
%%%%%%%%%%%%%%%%%%%%%%%%%%%%%%%%%%%%%
%\include{dedication}

% Start roman numbering
\pagenumbering{roman} \setcounter{page}{1}

%%%%%%%%%%%%%%%%%%%%%%%%%%%%%%%%
% SECTION : TABLE OF CONTENT
%%%%%%%%%%%%%%%%%%%%%%%%%%%%%%%%
\renewcommand{\contentsname}{Table of Contents}
\tableofcontents
\clearpage
\thispagestyle{empty}
\cleardoublepage

% Set to arabic numbering after this
\pagenumbering{arabic} \setcounter{page}{1}

%%%%%%%%%%%%%%%%%%%%%%%%%%%%%%%%
% SECTION : INTRODUCTION
%%%%%%%%%%%%%%%%%%%%%%%%%%%%%%%%
\chapter{Introduction}
\section{About}
This project's original motivation was to explore the study plans proposed by each section based on actual student choices -- analysing what course students took -- instead of the standard course prerequisites and descriptions that are generally available, and to offer a tool to allow anybody to take a look at the data for himself.

To do so required two important steps:
\begin{itemize}
\item Represent the students study history with a useful model;
\item Find good visual representations for the available data.
\end{itemize}

Both steps were achieved and solid foundations were lain to enable the implementation of new visualisations.

\section{Project Directory Structure}
We use sbt to build the code, and hence stick to its structure. Furthermore, the \verb|view|, \verb|core| and \verb|import| segments of the application each live in their own sbt sub-project.

The relevant directories are listed hereafter:

\dirtree{%
.0 .
.1 core/.
.2 src/main/.
.3 scala/.
.4 ch/epfl/recom.
.5 graph/.
.5 model/.
.6 administration/.
.5 processing/.
.6 maps/.
.7 assist/.
.5 storage/.
.6 db/.
.6 maps/.
.5 util/.
.3 plpgsql/.
.3 sqlview/.
.1 import/.
.2 src/main/scala/.
.3 ch/epfl/craft/recom/dimport/isa/.
.1 view/.
.2 src/main/scala.
.3 ch/epfl/craft/view.
.4 model/.
.4 snippet/.
.5 draw/.
.4 util/.
.4 view/.
}

\chapter{Model}
The model is built around the students and their course history, and this is reflected at the database level. Whenever going through the model is useless (most of the cases that were implemented), the data is accessed and processed directly through stored procedures.

For further in-depth data-mining where individual course histories must be analysed, we believe the model will prove useful.
\section{Structure}

\section{Database}

\subsection{Stored Procedures}
The following stored procedures are already available:
\paragraph{computeCostudents}
Populates the \verb|courserelationmap| table with the co-students count between any two courses.

\paragraph{sectionperTopicDetail} Gets details about section participation in each topic given by a section and during which academic level.

Example: \verb|SELECT sectionPerTopicDetail| ('{SHS}'::varchar[],'{MA1,MA3}'::varchar[],\\'2012-07-01','2012-07-01')

Returns: the topic isa\_id and name, the section participating in this topic (NOT the one teaching it),
the average participating students from the section and the total number of students in this topic.
these two last numbers are averaged over the number of courses that were considered in the query
(if a span greater that 1 semester is provided, several occurences of the same topic will be counted.) 
students not in the specified academic leves (here: MA1, MA3) are not counted.

\paragraph{sectionTopics} Returns the topics teached by the specified sections

usage: \verb|sectionTopics('{IN,SC}'::varchar[])|

\paragraph{sectionTopicsWStudentCount} Return the topics teached by the specified sections, and the quantity of students that attended during the specified period.

usage: \verb|sectionTopics('{IN,SC}'::varchar[],'2011-07-01','2012-07-01')|

\paragraph{topicCostudentsSemFilter} get the costudents quantity between any two topics taught by the specified sections during the specified time interval and at the specified academic level.

usage : select \verb|topicCostudents|('{SC,IN}'::varchar[],'{BA1}'::varchar[],
'2011-07-01','2012-07-01)

\paragraph{topicCostudents} Get the costudents quantity between two topics taught by the specified section during the specified time interval

usage : select topicCostudents('{SC,IN}'::varchar[],'2011-07-01','2012-07-01)

\paragraph{topicSectionRatio} Returns the sections present in this course and the corresponding ratio. The ratio is averaged over the number of courses that occurred in the given time interval.

usage: select \verb|topicsectionratio(<isa_id>,'2011-07-01','2012-07-01)|

\chapter{Visualisation}

\section{Technology}
The web application and data serving is done in Lift, which is also Scala based, enabling an easy integration with the core.

JavaScript is extensively used for the actual visualisation part in the browser.

\section{Implemented visualisations}


\end{document}
